\chapter{Design}

\section{Main Features}

The MMORPDND system at it's core relies on creating a D\&D universe. Within a D\&D universe are players, characters, stories, regions, maps, items, and more! Thereforethe MMORPDND system must accompany all of these. This project was initially conceived as a personal learning project and personal tool for managing D\&D campaigns (as a Dungeon Master tool). Because of this, much of this design was done AFTER many of the tools were already workin gand in development. This design will therefore correlate to that existing setup and explain some of the choices that were decided upon.

It is my hope that I can maintain good design documentation and outline thoughts in design decisions to be a helpful reference or helpful for anyone who chooses to make additions or improvements. At the time I've started writing this, many of the tools have already been created, but some are still only in their contemplation phase. 

\subsection{Mapping}

At the heart of MMORPDND lies very in depth and large maps (initially starting with the Matella region). These maps represents only portions of the starting planet and can therefore easily be expanded upon in terms of new regions, continents, etc. Each map is to be created in sufficiently high resolution so that creators can zoom in and expand upon the finer details of the area needed for a campaign.

For each campaign, a creator can select an area of the larger map, zoom into a smaller region, and develop the towns, cities, or surrounding areas that they will need for their campaign. Having the larger detailed map serves as a guide to keep locations and areas consistent as well as provide options for outside lore that is taking place simultaneously to the plaers campaign.

within a large world, there are typically many regions, cities, towns, rivers, mountain ranges, etc. Each of these can get lost and forgotten if they are not used. Within the MMORPDND system, all of these areas (once designated and designed) will have an HTML file generated for it. Upon running the main tools of the system, every named place will automatically be linked to for quick reference - providing any already created information to anyone who needs it.

\subsubsection{Wonderdraft and Dungeondraft}

Wonderdraft is a mapmaking tool that has many different features for building larger map areas such as regions, islands, cities, etc. It has a plethora of tools such as custom symbol additions, brushes, paths, etc. The tool creates clean looking maps that can be converted into different styles with the click of a button. One nice feature of the Wonderdraft tool is the ability to zoom into a region and create a more focused map from a larger one. For all these reasons (especially that last one), Wonderdraft was chosen as the intitial map making software for all MMORPDND maps and areas.

Dungeondraft is a close relative to Wonderdraft. It is designed primarily for smaller area's and building layouts. It is handy for making small battle maps and puzzle maps. It also has many unique features and supports custom symbols and objects. This was chosen as the MMORPDND battle map creator for its ease of use and large community containing many resources.

\subsection{Characters and NPCs}

A large part of D\&D is having characters and npc's. The MMORPDND system provides a quick and easy way to create nice looking NPC \textit{character sheets} (in the form of HTML pages) for quick reference or use. Many of the necessary parts of creating a character or NPC so that it is ready for gameplay can be automated. These include determining stats, health, bonuses, etc. The MMORPDND system can automate much of this process so that the user can create characters with only a fraction of the work normally needed.

It is often the case that NPCs have set locations, jobs, duties, names, families, or more. It's not always easy to keep track of these things. Within the MMORPDND system, each character will have a generated HTML page with all this information on it. The system will automatically search for an NPC name and automatically link all references to this page for quick reference!

\subsubsection{Other Game Glements}

Outside of maps and characters, there are many other elements to a D\&D game that can bring life to the world. These include items, lore, quests, factions, and much more. The MMORPDND system will manage all of these similar to how maps and characters are managed. With each different object gettin gits own HTML file to store information, the system will parse and link all files together appropriately.

\subsection{mmorpdnd.py}

This mmorpdnd.py script was the first tool created in the MMORPDND system. It was full created before any of this design documentation. It performs the bulk of the \textit{behind the scenes} automation such as file linking, formatting, etc. This tool is designed to do all the main features that makes the MMORPDND system complete and connected without any user input. The tool can be run via command line but also has an optional GUI that provides more specific options to run the individual components of the tool for time saving convenience.

Since the MMORPDND system relies on HTML files to store its data, there were a few considerations. since new files would be added constantly (as new data is created or added to the world), a mthod for keeping these files consistent was devised. The mmorpdnd.py script will use a template header and footer, scan for all html files (omitting some folders and files), and update these headers appropriately. This also makes modifying the headers and footers very simple as you only need to modify one file to then propogate the changes throughout the database.

It was decided that having an index file in each directory would be a convenient way of listing all files within that directory (html files, images, folders, etc). This process was also automated so that an index file is automatically created in each folder which has links to all desired sub-files. This makes navigating through the database simple.

Finally, the linking components were added to the mmorpdnd.py script. The idea behind this is to be able to quickly reference anything that is mentioned throughout the documents. The MMORPDND system will gather a compilation of all HTML file names. It then searches all the HTML files for references to those names. Upon finding them, the references are linked to the appropriate HTML files. It has been considered that this system will not function properly if multiple files with the same name are created. It is currently up to the creators to make sure file names are unique enough that this does not occur. As the world and system is still rather small, this is not an issue. A solution to this will likely be added as/if the system expands.

\subsection{creator.py}

The creator.py tool was the second tool in the MMORPDND system to be developed. This tool was also developed before these design documents. This tool is the one to take user-defined data and turn it into HTML files within the database. It was decided that simple text input would be best as input (as anyone will have access and ability to edit these) without any extra knowledge. These text files should be formatted properly for the system to understand the data being entered and properly turn it into the appropriate HTML files. 

Items of all sorts were created in a desired format that would appear appropriate and nice. These formats were then broken down into their base and common components. These compoents were turned into elements the user can define within the data to determine how it will appear (list, item, table, image, audio, etc). These are further elaborated on in the below sections and each has a simple and similar way of inputting data.

The \textit{character sheet} or character page is somewhat unique as it appears different than the other files. The .char file extension was used to generate these character files. This was the first type of input file to be created by the system. The user simply defines the needed parameters for a character and then the system wil fill in the rest and try to adapt appropriately given missing information. After this was completed, the approach turned to a more generic and versatile input type (.input files), which have much mroe flexibility. 

These generic input files for the system allow the user to have control over what types of data to input. These generic files will format text in blocks that look nice based on the type the user sets. For objects that have accompanying images (creatures, towns, maps, etc), a system was also devised to automatically detect multiple images. When naming images on linux or windows, the OS typically adds a number at the end of each similarly named file ('(1)', '(2)', ...). This is automatic on Windows and optional in Linux based machines. This format was chosen as it is easy for both operating systems to modify files in this way. When images have multiple files with these numbers, the system will automatically detect the images and display them all. 

The creator is setup and thoguth of as being the primary tool for new data/file creation. Therefore it is expected that new features will be added. It is a GUI application as it requires regular user input.

\subsection{Stockpile.py}

At the time of writing this document, this is currently the first undeveloped script/app that will have its design discussed. This application/script will be responsible for automating the S.T.O.C.K.P.I.L.E System (see section \ref{S.T.O.C.K.P.I.L.E}).



\section{Coding Choices}

Although many programming languages could have been selected for this project, Python was chosen for a few reasons. The primary reason was the familiarity and opportunity to further learn Python that was presented to the original creator. Another was it's simple to use and modify nature. With Python scripts, end users can open the code, make changes, and run them seemlessly without having to worry about proper setups and build systems (simple just having the correct python environment installed). Python provides easy to use string and file manipulation which was very relavent for the plans in this project. 


\section{Gameplay Features}

Since this system is ultimately for designing a D\&D universe, there are various new ideas relating to the gameplay itself that have also been developed alongside it. These include different styles of play, different ways to interact with characters, different managing techniques for Dungeon Masters, and more. I will attempt to outline some of these here but they may progress faster than this documentation does.

Personally, my largest hiccup when it comes to D\&D campaigns is finding people who have the time commitment. I do not necessarily mean once every week or even once every month, but rather willingness to meet one some regular cadence for an extended period of time. My personal DM style is to have a ton of depth and a slow overall story progression (not to be confused with a lack of action, adventure, or plot). I tend to plan for a campaign that will never end and let it play out as it does. The story will tend to change when players are not able to show up, but this is not always easy depending on where they left off. Unfortunately, scheduling is a nightmare and occasionally you can never find time to meet. Even though the players say they enjoy things well, there's always a player who drops out due to it taking too much time (even if you've only met 4 times in a year), which inevitably leaves to others dropping out too. To remedy this, I created what is known as the O.R.B.I.T System (outlined below), where players can come and go and not have to meet for every session. I have done something similar to this in the past and it worked out rather well, but this is a more robust version.

\subsection{O.R.B.I.T System}

One-shots, Risky Business, and Intriguing Tasks (O.R.B.I.T) is a system meticulously crafted to enhance the Dungeon Master's ability to run engaging campaigns in a flexible and dynamic manner. Tailored for Dungeons and Dragons (or other RPG) sessions, O.R.B.I.T. revolves around the idea of encapsulating complete adventures within a single session while seamlessly weaving an overarching narrative of covert operations, daring risks, and mysterious tasks.;Designed with the ever-changing availability of players in mind, O.R.B.I.T. allows for a revolving cast of characters, making it ideal for those seeking episodic gameplay with diverse and flexible party compositions. The acronym captures the essence of the campaign system: One-shots represent the self-contained missions or adventures. Risky Business involves the diverse array of financial and downtime activities and Intriguing Tasks form the backbone of an unfolding narrative that keeps players hooked and invested. In the game world, this system is embodied by the company Operations, Risky Business, and Intriguing Tasks, commonly known as Orbit. Orbit is a clandestine organization operating within Matella, masquerading as a reputable consultancy and security firm in the public eye. The city, rich with factions, secrets, and evolving events, serves as a dynamic backdrop to the adventures orchestrated by Orbit.

Whether players are pursuing high-stakes heists, unraveling political mysteries, or delving into ancient ruins, Orbit provides a modular and adaptable framework. Orbit Adventures shines in its flexibility, accommodating different playstyles and player schedules. The incorporation of downtime activities, risk management, and intriguing narratives creates a rich and immersive gaming experience. The game's structure empowers both Dungeon Masters and players to collaboratively shape a campaign that unfolds seamlessly, with each session contributing to the overarching story.

Between high-stakes missions, Orbit operatives engage in usualy life as well as some focused downtime activities to enhance their skills, manage their finances, and expand their inventory.

\begin{itemize}
\item Training for Advancement: The players can dedicate downtime to training and honing skills, facilitating character advancement. Training directly impacts the ability to level up, gaining new abilities, and improving existing ones. The Orbit system relies on the E.N.G.A.G.E system as the primary means of character enhancements for the players.

\item Financial Endeavors: The players (throught usual life activities) engage in various financial endeavors during downtime, including working regular jobs, making wise investments, or even taking calculated risks such as gambling. The outcomes directly affect the operatives' financial standing, influencing their resources for future missions. The Orbit system relies on the P.R.O.F.I.T system as the primary means of financial management for the players.

\item Inventory Management: The players can allocate time to inventory management, exploring markets, forging alliances, and discovering unique opportunities to acquire new items. This includes purchasing or crafting equipment and magical items, ensuring operatives are well-prepared for upcoming challenges. The Orbit system relies on the S.T.O.C.K.P.I.L.E Bazaar as the primary means of commerce interactions for the players.
\end{itemize}

Each operative's choices during downtime activities have a tangible impact on their character sheet. Whether it's gaining new skills, accumulating wealth, or expanding their arsenal, downtime is a crucial period where operatives invest in their personal growth and readiness for the next operation. Choose wisely, for the city's shadows hold both risks and rewards, and every decision shapes the trajectory of an operative's journey in Orbit Adventures.


\subsection{E.N.G.A.G.E. System}

The Earning New Gains through Active Growth Endeavors (E.N.G.A.G.E.) System is designed to infuse character progression with dynamic and active elements, allowing players to shape their characters through earned gains achieved via active growth endeavors that the characters live during game downtime. In this system, characters accumulate 'Gains' through successful sessions, missions, creative role-playing, and personal accomplishments. At the end of each session, players are rewarded with Gains, reflective of their active engagement and achievements. These points can be strategically allocated across various categories, such as leveling up, acquiring new skills, enhancing attributes, or unlocking special abilities.

The core philosophy of the system is the name of the system "Earning New Gains through Active Growth Endeavors," emphasizing that character development is not solely a passive process but a reflection of the character's active engagement and contributions to the unfolding narrative. Whether through mastering new skills, leveling up, or unlocking unique abilities, players have the agency to actively shape the trajectory of their characters' growth. Additionally, the E.N.G.A.G.E. System encourages players to participate actively in downtime activities, training, and quests, creating a synergistic relationship between in-game actions and character advancement. This dynamic approach ensures that characters evolve organically, reflecting the choices, actions, and endeavors of the players as they navigate the rich and immersive world of the campaign.

\subsection{P.R.O.F.I.T System}

In the Passive Returns and Opportunities for Financial Investment and Treasure (P.R.O.F.I.T) System (or simply P.R.O.F.I.T.S for short), characters have the opportunity to engage in various financial endeavors, each tier representing a different level of risk. This system is designed to be straightforward, providing characters with options to pursue low, medium, or high-risk financial activities during campaign downtime.

\begin{itemize}
\item Low Risk: Characters opt for stable employment, taking on a regular job in the city or performing services for a reliable employer. The type of employment can be anything but should be something fitting for the individual character. Characters receive a steady income, providing a reliable but moderate financial growth. Minimal loss potential, as the stability of the job ensures a consistent income, but the potential for significant wealth accumulation is limited.

\item Medium Risk: Characters decide to invest their funds in businesses, properties, or ventures that offer potential returns over time. Medium to high potential returns, with profits increasing as the investment matures. There's a chance of losing a portion or the entirety of the invested capital if the venture faces challenges or fails.

\item High Risk: Characters engage in high-stakes games of chance, placing bets or participating in gambling activities. High potential returns, with the possibility of substantial wealth gained through luck or skill. Significant loss potential, as characters risk losing their entire wager or investment in the unpredictable world of gambling.
\end{itemize}

At the end of each downtime period, characters declare their chosen financial activity (Low Risk, Medium Risk, or High Risk). The Dungeon Master determines the outcome based on the chosen tier, taking into account the associated risks and potential gains or losses. For simplicity, you can only pick one risk category per downtime period. A dice roll or a predetermined probability can be used to simulate the unpredictability of financial activities, especially in medium and high-risk scenarios. Characters can accumulate wealth over time, and their financial decisions may influence their lifestyle, ability to purchase items, or engage in other significant activities. The Financial Growth System aims to provide a balance between simplicity and engagement, offering characters diverse opportunities to navigate the financial landscape in the campaign. The specifics of this system are further detailed in the database files and will likely adapt and evolve over time.

\subsection{S.T.O.C.K.P.I.L.E System\label{S.T.O.C.K.P.I.L.E}}

Within the bustling market district, your characters will encounter a dynamic array of stores collectively referred to as the Special Treasures, Ordinances, Collectibles, Knickknacks, Paraphernalia, Items, Limited Ephemerals (S.T.O.C.K.P.I.L.E.) Bazaar. It's important to note that this term isn't assigned to a single establishment but is rather a convenient label used by the Dungeon Master.

To simplify and enrich your shopping experience, the Dungeon Master will manage a master list or spreadsheet encompassing all the items scattered across the diverse stores within the S.T.O.C.K.P.I.L.E. Bazaar. This central document acts as a comprehensive resource, allowing you to effortlessly browse through the available Special Treasures, Ordinances, Collectibles, Knickknacks, Paraphernalia, Items, and Limited Ephemerals.;During downtime, your characters are free to explore the various stores, each boasting its own unique offerings. The master list serves as an organized overview of all items accessible in the expansive market district. Prices are subject to fluctuation as stocks rise and fall, and items can seamlessly appear or disappear from the list based on availability.

It's crucial for players to manage their inventory and money diligently. Each item in the S.T.O.C.K.P.I.L.E. Bazaar will be accompanied by both a purchase price and a sell price. Players must keep track of their current money, as it serves as the sole limit to what they can purchase. As you navigate this vibrant market district, your choices in acquiring and selling items will directly impact your character's resources and potential for future acquisitions.;Embrace the excitement of discovery and bartering in the S.T.O.C.K.P.I.L.E. Bazaar, where each visit promises a unique journey through a world of wonders, and your astute financial decisions shape the course of your character's adventure.
