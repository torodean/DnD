%%Version Number
\newcommand{\Version}{1.002}

\renewcommand{\chaptername}{}

\titleformat{\chapter}[display]
{\normalfont\bfseries}{}{0pt}{\huge}

\setcounter{tocdepth}{3} % Set the depth of the table of contents
\setcounter{secnumdepth}{3} % Set the depth for numbering subsections and subsubsections

\geometry{margin=1in}

\setlength{\parindent}{25pt}

\renewcommand\bibname{References}

%%% Header and Footer Info
\pagestyle{fancy}
\fancyhead[LO]{\small {\textbf{MMORPDND Documentation - Version \Version}}}
\fancyhead[RE]{\small {\textbf{MMORPDND Documentation - Version \Version}}}
\fancyhead[C]{}
\fancyhead[RO]{\small \thepage}
\fancyhead[LE]{\small \thepage}
\fancyfoot[L]{}
\fancyfoot[C]{}
\fancyfoot[R]{}


%\setcounter{chapter}{-1} %starts the chapter labels at 0 instead of 1

\definecolor{backcolour}{rgb}{0.95,0.95,0.92}
\definecolor{commentcolour}{rgb}{.00,.245,.0}
\lstdefinestyle{python}{language=Python, tabsize=3, backgroundcolor=\color{backcolour},breaklines=true, basicstyle=\footnotesize, showstringspaces=false, commentstyle=\color{commentcolour}, keywordstyle=\color{blue}}
\lstset{style=python}

% Define a tcolorbox environment for code blocks
\tcbuselibrary{listings,skins,breakable}
\newtcblisting{codebox}[1][]{ % Add an optional argument for title
	listing only,
	colback=gray!10,
	colframe=gray!30,
	enhanced,
	breakable,
	boxrule=0.5pt,
	top=0pt,
	bottom=0pt,
	left=5pt,
	right=5pt,
	arc=0pt,
	outer arc=0pt,
	title={\textcolor{black}{\large #1}}, % Use the optional argument as the title
	listing options={
		basicstyle=\ttfamily\small,
		language=Python,
		showstringspaces=false,
		breaklines=true,
		numbers=left,
		numberstyle=\tiny,
		escapeinside={(*@}{@*)},
	},
}

% Define a tcolorbox environment for class code blocks
\newtcblisting{classbox}[1][]{ % Add an optional argument for title
	listing only,
	colback=blue!10, % Background color for classes
	colframe=blue!30, % Border color for classes
	enhanced,
	breakable,
	boxrule=0.5pt,
	top=0pt,
	bottom=0pt,
	left=5pt,
	right=5pt,
	arc=0pt,
	outer arc=0pt,
	title={\textcolor{black}{\large #1}}, % Use the optional argument as the title
	listing options={
		basicstyle=\ttfamily\small,
		language=Python,
		showstringspaces=false,
		breaklines=true,
		numbers=left,
		numberstyle=\tiny,
		escapeinside={(*@}{@*)},
	},
}



% Define colors
\definecolor{codebg}{RGB}{240,240,240}
\definecolor{codecomment}{RGB}{120,120,120}
\definecolor{codekeyword}{RGB}{0,0,255}
\definecolor{codestring}{RGB}{160,32,240}

% Define code style
\lstdefinestyle{mystyle}{
    backgroundcolor=\color{codebg},
    commentstyle=\color{codecomment},
    keywordstyle=\color{codekeyword},
    numberstyle=\tiny\color{codecomment},
    stringstyle=\color{codestring},
    breakatwhitespace=false,
    breaklines=true,
    captionpos=b,
    keepspaces=true,
    numbers=left,
    numbersep=5pt,
    showspaces=false,
    showstringspaces=false,
    showtabs=false,
    tabsize=2
}